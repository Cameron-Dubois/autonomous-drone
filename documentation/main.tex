\documentclass[11pt]{article}

\usepackage[margin=1in]{geometry}
\usepackage{setspace}
\usepackage{enumitem}
\usepackage{hyperref}

\setstretch{1.1}

\title{Autonomous Drone System Architecture}
\author{}
\date{}

\begin{document}

\maketitle

\section{Overview}

This document describes the overall system architecture for the autonomous drone project. The system is designed as a modular, layered architecture that separates hardware control, sensing, communication, vision, autonomy, and user interaction. This separation allows different subsystems to be developed, tested, and upgraded independently.

The system uses a hybrid communication model:
\begin{itemize}
    \item \textbf{Bluetooth Low Energy (BLE)} for control, telemetry, configuration, and low-latency signaling.
    \item \textbf{Wi-Fi} for high-bandwidth data such as camera images and video streams.
\end{itemize}

The architecture supports both manual operation and autonomous behavior, with the ability to scale from basic monitoring to advanced vision-based autonomy.

\section{System Components}

\subsection{On-Drone Components}

\begin{itemize}
    \item ESP32 microcontroller with attached camera module.
    \item Inertial Measurement Unit (IMU) for orientation and motion sensing.
    \item Power system including battery, voltage regulation, and monitoring.
    \item Optional sensors such as GPS, barometer, range sensors, or LiDAR.
    \item Optional flight controller (separate from ESP32) for stabilization and motor control.
\end{itemize}

\subsection{Ground Components}

\begin{itemize}
    \item Mobile application running on a smartphone.
    \item Optional external compute device for advanced processing (laptop, Raspberry Pi, Jetson).
\end{itemize}

\subsection{Supporting Infrastructure}

\begin{itemize}
    \item Version-controlled monorepo containing firmware, mobile app, documentation, and tooling.
\end{itemize}

\section{High-Level Architecture}

The system is divided into four major layers:
\begin{enumerate}
    \item Physical and Sensor Layer
    \item Embedded Control Layer
    \item Communication Layer
    \item Ground Control and Autonomy Layer
\end{enumerate}

Each layer has clearly defined responsibilities and interfaces.

\section{Physical and Sensor Layer}

This layer includes all physical hardware mounted on the drone.

\subsection*{Responsibilities}

\begin{itemize}
    \item Provide raw sensor data such as acceleration, angular velocity, position, altitude, and battery state.
    \item Capture visual data through the onboard camera.
    \item Deliver electrical power safely and reliably to all components.
\end{itemize}

This layer is hardware-specific and does not contain application logic.

\section{Embedded Control Layer (ESP32 Firmware)}

The ESP32 firmware acts as the primary embedded controller and hardware interface.

\subsection*{Responsibilities}

\begin{itemize}
    \item Initialize and manage all connected sensors.
    \item Manage camera operation and frame capture.
    \item Execute low-level state management and safety checks.
    \item Provide real-time telemetry describing the system state.
    \item Accept commands from external controllers and translate them into hardware actions.
    \item Interface with a separate flight controller if present (via UART or other serial protocol).
\end{itemize}

The firmware is designed to be event-driven and non-blocking to ensure responsiveness and stability.

\section{Communication Layer}

The communication layer defines how data moves between the drone and external systems.

\subsection{Bluetooth Low Energy (BLE)}

\begin{itemize}
    \item Used for command and control.
    \item Used for configuration and parameter updates.
    \item Used for telemetry notifications.
    \item Optimized for low bandwidth, low latency, and reliability.
\end{itemize}

\subsection{Wi-Fi}

\begin{itemize}
    \item Used for high-bandwidth data transfer.
    \item Primary transport for camera images and video streams.
    \item Supports HTTP and streaming protocols.
\end{itemize}

This dual-channel approach ensures that control and telemetry remain functional even when high-bandwidth camera data is active.

\section{Ground Control and Autonomy Layer}

This layer consists of software running off-drone.

\subsection{Mobile Application}

\begin{itemize}
    \item Acts as the primary user interface.
    \item Displays telemetry and system status.
    \item Displays live camera output.
    \item Allows manual control and mode selection.
    \item Sends configuration and mission commands to the drone.
\end{itemize}

\subsection{Autonomy and Vision Processing}

\begin{itemize}
    \item May run on the mobile device or an external compute unit.
    \item Processes camera data to extract environmental or target information.
    \item Generates high-level guidance commands rather than direct motor control.
    \item Supports behaviors such as object tracking, waypoint navigation, or obstacle avoidance.
\end{itemize}

\section{Control Flow}

\subsection{Manual Control Flow}

\begin{enumerate}
    \item User inputs commands in the mobile app.
    \item Commands are sent to the drone via BLE.
    \item ESP32 interprets commands and updates system state.
    \item Telemetry is sent back to the app via BLE.
\end{enumerate}

\subsection{Autonomous Control Flow}

\begin{enumerate}
    \item Camera data is streamed over Wi-Fi.
    \item Vision or autonomy logic processes the data.
    \item High-level commands are generated and sent to the ESP32.
    \item ESP32 enforces safety constraints and forwards commands to hardware or a flight controller.
\end{enumerate}

\section{Data Types}

\subsection{Control Data}

\begin{itemize}
    \item Mode changes
    \item Enable/disable commands
    \item Mission parameters
\end{itemize}

\subsection{Telemetry Data}

\begin{itemize}
    \item System state
    \item Battery voltage
    \item Sensor summaries
    \item Communication status
\end{itemize}

\subsection{Vision Data}

\begin{itemize}
    \item Still images
    \item Video streams
\end{itemize}

\subsection{Configuration Data}

\begin{itemize}
    \item Network settings
    \item Calibration parameters
    \item Control limits
\end{itemize}

\end{document}
